\title{
  (Mis)measuring competitiveness: the quantification of a malleable concept in the European Semester\thanks{
    The authors acknowledge funding by the Austrian Science Fund (FWF) 
    under grant number ZK 60-G27 and the Oesterreichische Nationalbank 
    (Austrian Central Bank, Anniversary Fund, project number: 18144).}
}

\author{
  \textbf{Claudius Gräbner-Radkowitsch} \\
  Institute for Socio-Economics\\
  University of Duisburg-Essen\\
  \& \\
  Institute for the Comprehensive Analysis of the Economy\\
  Johannes Kepler University Linz\\
  \href{mailto:claudius@claudius-graebner.com}{claudius@claudius-graebner.com} \\
  \and
  \textbf{Theresa Hager}\\
  Institute for the Comprehensive Analysis of the Economy\\
  Johannes Kepler University Linz\\
  \href{mailto:theresa.hager@jku.at}{theresa.hager@jku.at} \\
}
\pagenumbering{Roman}
\maketitle
\setcounter{footnote}{0}

\begin{abstract}
\small
This paper studies the conceptualization and quantification of 
`competitiveness’ within the main policy coordination framework 
of the EU, the European Semester. This topic warrants 
attention since `competitiveness' is not only of central importance in the European policy discourse, but also a theoretically ambiguous and malleable concept
with conflicting accentuations, all of which are subject of considerable 
academic and political debate. 
By investigating the translation of competition as a contested theoretical concept into 
concrete indicators within a legally binding document, the paper produces three main 
insights that deserve further attention, both scientifically and politically.
First, the indicators of the semester mainly measure \textit{cost} rather than 
\textit{technological} competitiveness, indicating a constriction of the 
concept at the operational level.
Second, while EU policy documents regularly stress the competitiveness of the 
European Union \textit{as a whole}, the indicators in the semester measure 
\textit{individual country competitiveness}.
Finally, the indicators in the Semester measure how the competitiveness of
single Member States changes over time, not how they perform relative to 
others. 
This shallows the heterogeneity of countries, which is problematic 
given recent findings according to which absolute differentials of
competitiveness across Member States is one important driver of accelerating 
polarization patterns in the Union.

\end{abstract}

\keywords{European Union \and competition \and performativity \and European semester
  \and political economy}
\newpage
\pagenumbering{arabic}
\setcounter{page}{1}

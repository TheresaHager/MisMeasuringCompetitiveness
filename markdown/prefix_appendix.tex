\renewcommand{\thepage}{A\arabic{page}}
\title{
  Appendix to\\ 
  (Mis)measuring competitiveness: the quantification of a malleable concept in the European Semester\thanks{
    The authors acknowledge funding by the Austrian Science Fund (FWF) 
    under grant number ZK 60-G27 and the Oesterreichische Nationalbank 
    (Austrian Central Bank, Anniversary Fund, project number: 18144).}
}

\author{
  \textbf{Claudius Gräbner-Radkowitsch} \\
  Institute for Socio-Economics\\
  University of Duisburg-Essen\\
  \& \\
  Institute for the Comprehensive Analysis of the Economy\\
  Johannes Kepler University Linz\\
  \href{mailto:claudius@claudius-graebner.com}{claudius@claudius-graebner.com} \\
  \and
  \textbf{Theresa Hager}\\
  Institute for the Comprehensive Analysis of the Economy\\
  Johannes Kepler University Linz\\
  \href{mailto:theresa.hager@jku.at}{theresa.hager@jku.at} \\
}
\maketitle
\setcounter{footnote}{0}
\begin{abstract}
\small
This appendix complements the main paper by providing additional information
on the European Semester and alternative ways to illustrate some of the main
findings. Section \ref{sec:rules} contains a detailed description of the 
content and the time line of the Semester.
Section \ref{sec:descriptives} provides information about how Member States 
were classified according to the MIP and SGP criteria.
Then, Section \ref{sec:pics} reproduces Figure 3 of the main paper for the
SGP and MIP categories discussed before, as well as for individual Member States.
Finally, Section \ref{sec:is} summarizes information about the 
country-specific recommendations, most importantly the policy areas covered and 
assessment of the degree to which the recommendations were implemented by the
Member States.

\end{abstract}

\keywords{European Union \and competition \and performativity \and European semester \and political economy }
\newpage
